\section{Workflow}
In this section, we review the methods that have been used to perform kNN join on MapReduce. Overall, they all share the same 
workflow, comprised of three stages: (i)~data preprocessing (ii)~data partitioning and organization 
and (iii)~computation of kNN. We will focus on the 
following works: the initial (basic, not optimized) idea, which we 
call H-BkNNJ 
%\TODO{why this name? --B stands for basic--}
, and its improvements H-BNLJ (Block Nested Loop Join) and H-BRJ (Block Nested R-Tree Join) in \cite{Zhang:2012:EPK:2247596.2247602}, as well as more advanced 
solutions such as 
H-$z$kNNJ (z-value) in \cite{Zhang:2012:EPK:2247596.2247602}, RankReduce (Locality Sensitive Hashing) in \cite{Stupar10rankreduce-} and PGBJ (Voronoi) in \cite{Lu:2012:EPK:
2336664.2336674}. 

%\TODO{Check/rewrite --checked--}

%
%these two methods do not need any pre-processing or data organization 
%strategies. Additional, some advanced methods which invoke some pre-processing and data organization strategies such as H-
%zkNNJ \cite{Zhang:2012:EPK:2247596.2247602} and H-VDkNNJ \cite{Lu:2012:EPK:2336664.2336674} will also be presented.

%<<<<<<< .mine

% =======
%We review here the methods that have been used to perform kNN join on MapReduce. Here we mainly present the methods below: the basic idea (which we 
%call H-BkNNJ) and its improvement H-BNLJ \cite{Zhang:2012:EPK:2247596.2247602}, these two methods do not need any pre-processing or data organization 
%strategies. Additional, some advanced methods which invoke some pre-processing and data organization strategies such as H-zkNNJ \cite{Zhang:2012:EPK:
%2247596.2247602} and H-VDkNNJ \cite{Lu:2012:EPK:2336664.2336674} will also be presented.
%
%Totally, to process these methods on MapReduce we need 3 stages: (i) Data Pre-Processing (ii) Data Organization and (iii) Execution. We will present 
%these stages separately.>>>>>>> .r2824
