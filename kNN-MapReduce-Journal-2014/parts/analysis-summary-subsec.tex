
\subsection{Wrap up}
Although the workflow for computing kNN on MapReduce is the same for all existing solutions, the guarantees offered by each of them vary a lot. As load 
balancing is a key point to reduce completion time, one should carefully choose the partitioning method to achieve this goal. Also, the accuracy of the 
computing system is crucial: are exact results really needed? If not, then one might trade accuracy for efficiency, by using data transformation 
techniques before the actual computation. 
Complexity of the global system should also be taken into account for particular needs, although it is often related to the accuracy: an exact system is 
usually more complex than an approximate one.
Table~\ref{summary_table_new} shows a summary of the systems we have examined and their main characteristics.

Due to the multiple parameters and very different steps for each algorithm, we had to limit our complexity analysis to 
common operations. Moreover, for some of them, the complexity depends on parameters set by a user or some properties
of the dataset. Therefore, the total processing time might be different, in practice, than the one predicted by the 
theoretical analysis. That is why it is important to have a thorough experimental study.


%In this section, we analyze existing works from another point of view than the workflow they follow. We first analyze load balancing to figure 
%out the impact of the partitioning strategy. We then point out solutions that produce an exact result from solutions that produce an approximate 
%result, in order to complete faster. We also compare the expected complexity of the systems. Finally, we consider the problem of changes in 
%the dataset (either in $R$ or $S$) and the challenges it raises. 
